% Options for packages loaded elsewhere
\PassOptionsToPackage{unicode}{hyperref}
\PassOptionsToPackage{hyphens}{url}
%
\documentclass[
]{article}
\usepackage{amsmath,amssymb}
\usepackage{lmodern}
\usepackage{ifxetex,ifluatex}
\ifnum 0\ifxetex 1\fi\ifluatex 1\fi=0 % if pdftex
  \usepackage[T1]{fontenc}
  \usepackage[utf8]{inputenc}
  \usepackage{textcomp} % provide euro and other symbols
\else % if luatex or xetex
  \usepackage{unicode-math}
  \defaultfontfeatures{Scale=MatchLowercase}
  \defaultfontfeatures[\rmfamily]{Ligatures=TeX,Scale=1}
\fi
% Use upquote if available, for straight quotes in verbatim environments
\IfFileExists{upquote.sty}{\usepackage{upquote}}{}
\IfFileExists{microtype.sty}{% use microtype if available
  \usepackage[]{microtype}
  \UseMicrotypeSet[protrusion]{basicmath} % disable protrusion for tt fonts
}{}
\makeatletter
\@ifundefined{KOMAClassName}{% if non-KOMA class
  \IfFileExists{parskip.sty}{%
    \usepackage{parskip}
  }{% else
    \setlength{\parindent}{0pt}
    \setlength{\parskip}{6pt plus 2pt minus 1pt}}
}{% if KOMA class
  \KOMAoptions{parskip=half}}
\makeatother
\usepackage{xcolor}
\IfFileExists{xurl.sty}{\usepackage{xurl}}{} % add URL line breaks if available
\IfFileExists{bookmark.sty}{\usepackage{bookmark}}{\usepackage{hyperref}}
\hypersetup{
  pdftitle={PS2},
  pdfauthor={Doron Zamir},
  hidelinks,
  pdfcreator={LaTeX via pandoc}}
\urlstyle{same} % disable monospaced font for URLs
\usepackage[margin=1in]{geometry}
\usepackage{color}
\usepackage{fancyvrb}
\newcommand{\VerbBar}{|}
\newcommand{\VERB}{\Verb[commandchars=\\\{\}]}
\DefineVerbatimEnvironment{Highlighting}{Verbatim}{commandchars=\\\{\}}
% Add ',fontsize=\small' for more characters per line
\usepackage{framed}
\definecolor{shadecolor}{RGB}{248,248,248}
\newenvironment{Shaded}{\begin{snugshade}}{\end{snugshade}}
\newcommand{\AlertTok}[1]{\textcolor[rgb]{0.94,0.16,0.16}{#1}}
\newcommand{\AnnotationTok}[1]{\textcolor[rgb]{0.56,0.35,0.01}{\textbf{\textit{#1}}}}
\newcommand{\AttributeTok}[1]{\textcolor[rgb]{0.77,0.63,0.00}{#1}}
\newcommand{\BaseNTok}[1]{\textcolor[rgb]{0.00,0.00,0.81}{#1}}
\newcommand{\BuiltInTok}[1]{#1}
\newcommand{\CharTok}[1]{\textcolor[rgb]{0.31,0.60,0.02}{#1}}
\newcommand{\CommentTok}[1]{\textcolor[rgb]{0.56,0.35,0.01}{\textit{#1}}}
\newcommand{\CommentVarTok}[1]{\textcolor[rgb]{0.56,0.35,0.01}{\textbf{\textit{#1}}}}
\newcommand{\ConstantTok}[1]{\textcolor[rgb]{0.00,0.00,0.00}{#1}}
\newcommand{\ControlFlowTok}[1]{\textcolor[rgb]{0.13,0.29,0.53}{\textbf{#1}}}
\newcommand{\DataTypeTok}[1]{\textcolor[rgb]{0.13,0.29,0.53}{#1}}
\newcommand{\DecValTok}[1]{\textcolor[rgb]{0.00,0.00,0.81}{#1}}
\newcommand{\DocumentationTok}[1]{\textcolor[rgb]{0.56,0.35,0.01}{\textbf{\textit{#1}}}}
\newcommand{\ErrorTok}[1]{\textcolor[rgb]{0.64,0.00,0.00}{\textbf{#1}}}
\newcommand{\ExtensionTok}[1]{#1}
\newcommand{\FloatTok}[1]{\textcolor[rgb]{0.00,0.00,0.81}{#1}}
\newcommand{\FunctionTok}[1]{\textcolor[rgb]{0.00,0.00,0.00}{#1}}
\newcommand{\ImportTok}[1]{#1}
\newcommand{\InformationTok}[1]{\textcolor[rgb]{0.56,0.35,0.01}{\textbf{\textit{#1}}}}
\newcommand{\KeywordTok}[1]{\textcolor[rgb]{0.13,0.29,0.53}{\textbf{#1}}}
\newcommand{\NormalTok}[1]{#1}
\newcommand{\OperatorTok}[1]{\textcolor[rgb]{0.81,0.36,0.00}{\textbf{#1}}}
\newcommand{\OtherTok}[1]{\textcolor[rgb]{0.56,0.35,0.01}{#1}}
\newcommand{\PreprocessorTok}[1]{\textcolor[rgb]{0.56,0.35,0.01}{\textit{#1}}}
\newcommand{\RegionMarkerTok}[1]{#1}
\newcommand{\SpecialCharTok}[1]{\textcolor[rgb]{0.00,0.00,0.00}{#1}}
\newcommand{\SpecialStringTok}[1]{\textcolor[rgb]{0.31,0.60,0.02}{#1}}
\newcommand{\StringTok}[1]{\textcolor[rgb]{0.31,0.60,0.02}{#1}}
\newcommand{\VariableTok}[1]{\textcolor[rgb]{0.00,0.00,0.00}{#1}}
\newcommand{\VerbatimStringTok}[1]{\textcolor[rgb]{0.31,0.60,0.02}{#1}}
\newcommand{\WarningTok}[1]{\textcolor[rgb]{0.56,0.35,0.01}{\textbf{\textit{#1}}}}
\usepackage{graphicx}
\makeatletter
\def\maxwidth{\ifdim\Gin@nat@width>\linewidth\linewidth\else\Gin@nat@width\fi}
\def\maxheight{\ifdim\Gin@nat@height>\textheight\textheight\else\Gin@nat@height\fi}
\makeatother
% Scale images if necessary, so that they will not overflow the page
% margins by default, and it is still possible to overwrite the defaults
% using explicit options in \includegraphics[width, height, ...]{}
\setkeys{Gin}{width=\maxwidth,height=\maxheight,keepaspectratio}
% Set default figure placement to htbp
\makeatletter
\def\fps@figure{htbp}
\makeatother
\setlength{\emergencystretch}{3em} % prevent overfull lines
\providecommand{\tightlist}{%
  \setlength{\itemsep}{0pt}\setlength{\parskip}{0pt}}
\setcounter{secnumdepth}{-\maxdimen} % remove section numbering
\ifluatex
  \usepackage{selnolig}  % disable illegal ligatures
\fi

\title{PS2}
\usepackage{etoolbox}
\makeatletter
\providecommand{\subtitle}[1]{% add subtitle to \maketitle
  \apptocmd{\@title}{\par {\large #1 \par}}{}{}
}
\makeatother
\subtitle{ml4econ, HUJI 2021}
\author{Doron Zamir}
\date{4/26/2021}

\begin{document}
\maketitle

\hypertarget{load-packages}{%
\section{Load Packages}\label{load-packages}}

\begin{Shaded}
\begin{Highlighting}[]
\ControlFlowTok{if}\NormalTok{ (}\SpecialCharTok{!}\FunctionTok{require}\NormalTok{(}\StringTok{"pacman"}\NormalTok{)) }\FunctionTok{install.packages}\NormalTok{(}\StringTok{"pacman"}\NormalTok{)}

\NormalTok{pacman}\SpecialCharTok{::}\FunctionTok{p\_load}\NormalTok{(}
\NormalTok{  tidyverse,}
\NormalTok{  tidymodels,}
\NormalTok{  vip,}
\NormalTok{  here,}
\NormalTok{  readxl,}
\NormalTok{  DataExplorer,}
\NormalTok{  caret,}
\NormalTok{  glmnet}
\NormalTok{)}
\end{Highlighting}
\end{Shaded}

set seed:

\begin{Shaded}
\begin{Highlighting}[]
\FunctionTok{set.seed}\NormalTok{(}\DecValTok{100}\NormalTok{)}
\end{Highlighting}
\end{Shaded}

\hypertarget{linear-regression}{%
\section{Linear regression}\label{linear-regression}}

\hypertarget{can-we-use-the-data-for-prediction-without-assumptions-why}{%
\subsubsection{\texorpdfstring{1. \emph{Can we use the data for
prediction without assumptions?
Why?}}{1. Can we use the data for prediction without assumptions? Why?}}\label{can-we-use-the-data-for-prediction-without-assumptions-why}}

Since we are not interested in the casual effect of the \texttt{x}'s,
nor the underlying model related to the DGP, the only assumption we need
to make in order to make predictions is that of a \textbf{Stable DGP}.

\hypertarget{what-is-the-downside-in-adding-interactions}{%
\subsubsection{\texorpdfstring{2. \emph{What is the downside in adding
interactions?
}}{2. What is the downside in adding interactions? }}\label{what-is-the-downside-in-adding-interactions}}

When adding interactions, we are increasing the number of
featuers(\texttt{x}'s) in our model. The \(k\) in \(n/k\) ratio
increases and our model becomes more saturated. This might cause to
overfitting of the model.

\hypertarget{why-is-eux-sim-n0sigma-assumptions-strong}{%
\subsubsection{\texorpdfstring{3. \emph{Why is
\(E(u|X) \sim N(0,\sigma)\) assumptions
strong?}}{3. Why is E(u\textbar X) \textbackslash sim N(0,\textbackslash sigma) assumptions strong?}}\label{why-is-eux-sim-n0sigma-assumptions-strong}}

This assumptions are strong: * although it could be the case that the
residual converges in distribution to th Normal distribution (from CLT),
it's not the only option: for example, if we observe the frequency that
an event occurs, it's more likley that it will be from the Poisson
distribution.

\begin{itemize}
\tightlist
\item
  it might not be the case that the error term has a mean of zero, but
  you can always ajust it to be zero by changing \(\beta_0\).
\end{itemize}

\hypertarget{the-betas-ci}{%
\subsubsection{\texorpdfstring{4. \emph{The \(\beta\)s
CI:}}{4. The \textbackslash betas CI:}}\label{the-betas-ci}}

From the above assumptions on the distribution of the error term, we can
conclude the distribution of the \(\beta\)s, and by using the sample
standart errors as an estimatior to it's SD, we can calculate the CI.

\hypertarget{emprical-exercise---the-wine-dataset}{%
\subsection{Emprical Exercise - The Wine
Dataset}\label{emprical-exercise---the-wine-dataset}}

\hypertarget{loading-the-data}{%
\subsubsection{Loading the Data:}\label{loading-the-data}}

\begin{Shaded}
\begin{Highlighting}[]
\NormalTok{wine\_raw }\OtherTok{\textless{}{-}} 
  \FunctionTok{here}\NormalTok{(}\StringTok{"Data"}\NormalTok{,}\StringTok{"winequality\_red.csv"}\NormalTok{) }\SpecialCharTok{\%\textgreater{}\%} 
  \FunctionTok{read\_csv}\NormalTok{()}
\end{Highlighting}
\end{Shaded}

\hypertarget{exploring-the-data}{%
\subsubsection{Exploring the data}\label{exploring-the-data}}

Note the data has no \texttt{na}.

plot continuous variables:

\begin{Shaded}
\begin{Highlighting}[]
\NormalTok{wine\_raw }\SpecialCharTok{\%\textgreater{}\%} 
  \FunctionTok{plot\_histogram}\NormalTok{()}
\end{Highlighting}
\end{Shaded}

\includegraphics{ps_2_files/figure-latex/unnamed-chunk-4-1..svg}

plot variables against \texttt{quality}:

\begin{Shaded}
\begin{Highlighting}[]
\NormalTok{wine\_raw }\SpecialCharTok{\%\textgreater{}\%} 
  \FunctionTok{plot\_boxplot}\NormalTok{(}\AttributeTok{by =} \StringTok{"quality"}\NormalTok{)}
\end{Highlighting}
\end{Shaded}

\includegraphics{ps_2_files/figure-latex/unnamed-chunk-5-1..svg}

\hypertarget{model}{%
\subsubsection{Model}\label{model}}

train/test split (70/30):

\begin{Shaded}
\begin{Highlighting}[]
\NormalTok{wine\_split }\OtherTok{\textless{}{-}}\NormalTok{ wine\_raw }\SpecialCharTok{\%\textgreater{}\%} \FunctionTok{initial\_split}\NormalTok{(}\AttributeTok{prop =} \FloatTok{0.7}\NormalTok{)}
\NormalTok{wine\_train }\OtherTok{\textless{}{-}}\NormalTok{ wine\_split }\SpecialCharTok{\%\textgreater{}\%} \FunctionTok{training}\NormalTok{()}
\NormalTok{wine\_test }\OtherTok{\textless{}{-}}\NormalTok{ wine\_split }\SpecialCharTok{\%\textgreater{}\%} \FunctionTok{testing}\NormalTok{()}
\end{Highlighting}
\end{Shaded}

Simple Linear Model:

\begin{Shaded}
\begin{Highlighting}[]
\NormalTok{wine\_lm }\OtherTok{\textless{}{-}}\NormalTok{ wine\_train }\SpecialCharTok{\%\textgreater{}\%} 
  \FunctionTok{lm}\NormalTok{(quality}\SpecialCharTok{\textasciitilde{}}\NormalTok{.,}
     \AttributeTok{data =}\NormalTok{ .)}

\NormalTok{wine\_lm }\SpecialCharTok{\%\textgreater{}\%} \FunctionTok{tidy}\NormalTok{()}
\end{Highlighting}
\end{Shaded}

\begin{verbatim}
## # A tibble: 12 x 5
##    term                   estimate std.error statistic  p.value
##    <chr>                     <dbl>     <dbl>     <dbl>    <dbl>
##  1 (Intercept)             8.82    26.5          0.333 7.39e- 1
##  2 `fixed acidity`         0.0201   0.0322       0.625 5.32e- 1
##  3 `volatile acidity`     -1.19     0.153       -7.74  2.30e-14
##  4 `citric acid`          -0.246    0.184       -1.34  1.82e- 1
##  5 `residual sugar`        0.0177   0.0197       0.899 3.69e- 1
##  6 chlorides              -2.30     0.545       -4.21  2.76e- 5
##  7 `free sulfur dioxide`   0.00396  0.00273      1.45  1.47e- 1
##  8 `total sulfur dioxide` -0.00291  0.000876    -3.32  9.32e- 4
##  9 density                -4.35    27.0         -0.161 8.72e- 1
## 10 pH                     -0.483    0.239       -2.02  4.36e- 2
## 11 sulphates               0.988    0.144        6.87  1.10e-11
## 12 alcohol                 0.274    0.0332       8.24  4.83e-16
\end{verbatim}

Prediction:

\begin{Shaded}
\begin{Highlighting}[]
\NormalTok{wine\_test\_pred }\OtherTok{\textless{}{-}}\NormalTok{ wine\_test }\SpecialCharTok{\%\textgreater{}\%} 
  \FunctionTok{mutate}\NormalTok{(}\AttributeTok{pred =}\FunctionTok{predict}\NormalTok{(}
\NormalTok{    wine\_lm,}
    \AttributeTok{newdata =}\NormalTok{ wine\_test)) }\SpecialCharTok{\%\textgreater{}\%} 
  \FunctionTok{mutate}\NormalTok{(}\AttributeTok{pred =} \FunctionTok{round}\NormalTok{(pred,}\AttributeTok{digits =} \DecValTok{0}\NormalTok{))}

\NormalTok{wine\_test\_pred }\SpecialCharTok{\%\textgreater{}\%} \FunctionTok{select}\NormalTok{(pred) }\SpecialCharTok{\%\textgreater{}\%} \FunctionTok{sample\_n}\NormalTok{(}\DecValTok{6}\NormalTok{)}
\end{Highlighting}
\end{Shaded}

\begin{verbatim}
## # A tibble: 6 x 1
##    pred
##   <dbl>
## 1     6
## 2     5
## 3     6
## 4     5
## 5     6
## 6     5
\end{verbatim}

calculate metrics:

\begin{Shaded}
\begin{Highlighting}[]
\NormalTok{wine\_test\_pred }\SpecialCharTok{\%\textgreater{}\%} 
  \FunctionTok{summarise}\NormalTok{(}
    \AttributeTok{RMSE =} \FunctionTok{mean}\NormalTok{((pred }\SpecialCharTok{{-}}\NormalTok{ quality)}\SpecialCharTok{\^{}}\DecValTok{2}\NormalTok{)}\SpecialCharTok{\^{}}\FloatTok{0.5}\NormalTok{,}
    \StringTok{"R\^{}2"} \OtherTok{=} \FunctionTok{cov}\NormalTok{(pred,quality)}\SpecialCharTok{\^{}}\DecValTok{2}\NormalTok{,}
    \AttributeTok{MAE =} \FunctionTok{mean}\NormalTok{(}\FunctionTok{abs}\NormalTok{(pred}\SpecialCharTok{{-}}\NormalTok{quality))}
\NormalTok{)}
\end{Highlighting}
\end{Shaded}

\begin{verbatim}
## # A tibble: 1 x 3
##    RMSE  `R^2`   MAE
##   <dbl>  <dbl> <dbl>
## 1 0.645 0.0628 0.386
\end{verbatim}

\hypertarget{rmse-vs.-t-test}{%
\subsubsection{RMSE vs.~t-test}\label{rmse-vs.-t-test}}

while t test checks for significant of the \(\beta\)s, the \texttt{RMSE}
checks for the accuracy of the entire model.

\hypertarget{emprical-exercise---the-heart-dataset}{%
\section{Emprical Exercise - The Heart
Dataset}\label{emprical-exercise---the-heart-dataset}}

\hypertarget{can-we-use-linear-regression-for-binary-outcomes}{%
\subsubsection{\texorpdfstring{\emph{Can we use linear regression for
binary
outcomes?}}{Can we use linear regression for binary outcomes?}}\label{can-we-use-linear-regression-for-binary-outcomes}}

Yes, we can use binary regression model (i.e., OLS when the \(Y\) is
binary). We might accure some problems, as the predictions are not
limtied to the \([0,1]\) segment. It is more common to use
\texttt{probit/logit}. \#\#\# Loading the Data:

\begin{Shaded}
\begin{Highlighting}[]
\NormalTok{heart\_raw }\OtherTok{\textless{}{-}} 
  \FunctionTok{here}\NormalTok{(}\StringTok{"Data"}\NormalTok{,}\StringTok{"heart.csv"}\NormalTok{) }\SpecialCharTok{\%\textgreater{}\%} 
  \FunctionTok{read\_csv}\NormalTok{()}

\NormalTok{heart\_raw }\SpecialCharTok{\%\textgreater{}\%} \FunctionTok{head}\NormalTok{()}
\end{Highlighting}
\end{Shaded}

\begin{verbatim}
## # A tibble: 6 x 14
##     age   sex    cp trestbps  chol   fbs restecg thalach exang oldpeak slope
##   <dbl> <dbl> <dbl>    <dbl> <dbl> <dbl>   <dbl>   <dbl> <dbl>   <dbl> <dbl>
## 1    63     1     3      145   233     1       0     150     0     2.3     0
## 2    37     1     2      130   250     0       1     187     0     3.5     0
## 3    41     0     1      130   204     0       0     172     0     1.4     2
## 4    56     1     1      120   236     0       1     178     0     0.8     2
## 5    57     0     0      120   354     0       1     163     1     0.6     2
## 6    57     1     0      140   192     0       1     148     0     0.4     1
## # ... with 3 more variables: ca <dbl>, thal <dbl>, target <dbl>
\end{verbatim}

\hypertarget{exploring-the-data-1}{%
\subsubsection{Exploring the data}\label{exploring-the-data-1}}

Note the data has no \texttt{na}.

plot continuous variables:

\begin{Shaded}
\begin{Highlighting}[]
\NormalTok{heart\_raw }\SpecialCharTok{\%\textgreater{}\%} 
  \FunctionTok{plot\_histogram}\NormalTok{()}
\end{Highlighting}
\end{Shaded}

\includegraphics{ps_2_files/figure-latex/unnamed-chunk-11-1..svg} \#\#\#
Model

train/test split (70/30):

\begin{Shaded}
\begin{Highlighting}[]
\NormalTok{heart\_split }\OtherTok{\textless{}{-}}\NormalTok{ heart\_raw }\SpecialCharTok{\%\textgreater{}\%} \FunctionTok{initial\_split}\NormalTok{(}\AttributeTok{prop =} \FloatTok{0.7}\NormalTok{)}
\NormalTok{heart\_train }\OtherTok{\textless{}{-}}\NormalTok{ heart\_split }\SpecialCharTok{\%\textgreater{}\%} \FunctionTok{training}\NormalTok{()}
\NormalTok{heart\_test }\OtherTok{\textless{}{-}}\NormalTok{ heart\_split }\SpecialCharTok{\%\textgreater{}\%} \FunctionTok{testing}\NormalTok{()}
\end{Highlighting}
\end{Shaded}

\hypertarget{linear-regression-1}{%
\paragraph{Linear regression:}\label{linear-regression-1}}

\begin{Shaded}
\begin{Highlighting}[]
\NormalTok{heart\_lm }\OtherTok{\textless{}{-}}  \FunctionTok{lm}\NormalTok{(target }\SpecialCharTok{\textasciitilde{}}\NormalTok{.,}
                \AttributeTok{data =}\NormalTok{heart\_train)}

\NormalTok{heart\_lm }\SpecialCharTok{\%\textgreater{}\%} \FunctionTok{tidy}\NormalTok{()}
\end{Highlighting}
\end{Shaded}

\begin{verbatim}
## # A tibble: 14 x 5
##    term         estimate std.error statistic   p.value
##    <chr>           <dbl>     <dbl>     <dbl>     <dbl>
##  1 (Intercept)  0.757     0.359        2.11  0.0361   
##  2 age         -0.00105   0.00321     -0.327 0.744    
##  3 sex         -0.231     0.0579      -3.99  0.0000929
##  4 cp           0.115     0.0272       4.22  0.0000365
##  5 trestbps    -0.00157   0.00158     -0.990 0.323    
##  6 chol        -0.000798  0.000539    -1.48  0.140    
##  7 fbs          0.0110    0.0717       0.154 0.878    
##  8 restecg      0.0364    0.0481       0.757 0.450    
##  9 thalach      0.00440   0.00144      3.06  0.00249  
## 10 exang       -0.143     0.0614      -2.32  0.0213   
## 11 oldpeak     -0.0609    0.0304      -2.00  0.0464   
## 12 slope        0.0356    0.0503       0.707 0.480    
## 13 ca          -0.0739    0.0258      -2.86  0.00462  
## 14 thal        -0.119     0.0436      -2.73  0.00693
\end{verbatim}

predict:

\begin{Shaded}
\begin{Highlighting}[]
\NormalTok{p }\OtherTok{\textless{}{-}} \FloatTok{0.9}
\NormalTok{heart\_test\_pred }\OtherTok{\textless{}{-}}\NormalTok{ heart\_test }\SpecialCharTok{\%\textgreater{}\%} 
  \FunctionTok{mutate}\NormalTok{(}
    \AttributeTok{pred =} \FunctionTok{predict}\NormalTok{(heart\_lm,heart\_test,}\AttributeTok{type=}\StringTok{"response"}\NormalTok{),}
    \AttributeTok{target\_fct =} \FunctionTok{factor}\NormalTok{(target),}
    \AttributeTok{pred\_class =} \FunctionTok{if\_else}\NormalTok{(pred }\SpecialCharTok{\textgreater{}}\NormalTok{ p,}\DecValTok{1}\NormalTok{,}\DecValTok{0}\NormalTok{),}
    \AttributeTok{pred\_class =} \FunctionTok{factor}\NormalTok{(pred\_class))}

\NormalTok{heart\_test\_pred}\SpecialCharTok{$}\NormalTok{pred }\SpecialCharTok{\%\textgreater{}\%} \FunctionTok{max}\NormalTok{()}
\end{Highlighting}
\end{Shaded}

\begin{verbatim}
## [1] 1.226172
\end{verbatim}

\begin{Shaded}
\begin{Highlighting}[]
\NormalTok{heart\_test\_pred}\SpecialCharTok{$}\NormalTok{pred }\SpecialCharTok{\%\textgreater{}\%} \FunctionTok{min}\NormalTok{()}
\end{Highlighting}
\end{Shaded}

\begin{verbatim}
## [1] -0.3533657
\end{verbatim}

It's easy to se the prediction are not in {[}0,1{]}.

ROC curve

\begin{Shaded}
\begin{Highlighting}[]
\NormalTok{heart\_test\_pred }\SpecialCharTok{\%\textgreater{}\%}
  \FunctionTok{roc\_curve}\NormalTok{(target\_fct,pred) }\SpecialCharTok{\%\textgreater{}\%} \FunctionTok{autoplot}\NormalTok{()}
\end{Highlighting}
\end{Shaded}

\includegraphics{ps_2_files/figure-latex/unnamed-chunk-15-1..svg}

\begin{Shaded}
\begin{Highlighting}[]
\NormalTok{ heart\_test\_pred  }\SpecialCharTok{\%\textgreater{}\%}
  \FunctionTok{conf\_mat}\NormalTok{(target\_fct,pred\_class)}
\end{Highlighting}
\end{Shaded}

\begin{verbatim}
##           Truth
## Prediction  0  1
##          0 42 33
##          1  1 14
\end{verbatim}

The model is not good, at all\ldots{}

\hypertarget{logit}{%
\paragraph{Logit}\label{logit}}

estimating logistic model:

\begin{Shaded}
\begin{Highlighting}[]
\NormalTok{heart\_logit }\OtherTok{\textless{}{-}} \FunctionTok{glm}\NormalTok{(}\AttributeTok{formula =}\NormalTok{ target }\SpecialCharTok{\textasciitilde{}}\NormalTok{.,}
                   \AttributeTok{data =}\NormalTok{ heart\_train,}
                   \AttributeTok{family =} \StringTok{"binomial"}\NormalTok{)}

\NormalTok{heart\_logit }\SpecialCharTok{\%\textgreater{}\%} \FunctionTok{summary}\NormalTok{()}
\end{Highlighting}
\end{Shaded}

\begin{verbatim}
## 
## Call:
## glm(formula = target ~ ., family = "binomial", data = heart_train)
## 
## Deviance Residuals: 
##     Min       1Q   Median       3Q      Max  
## -2.3655  -0.3939   0.1532   0.5855   2.8860  
## 
## Coefficients:
##              Estimate Std. Error z value Pr(>|z|)    
## (Intercept)  2.935738   3.017185   0.973  0.33055    
## age         -0.010563   0.027693  -0.381  0.70287    
## sex         -2.076254   0.554251  -3.746  0.00018 ***
## cp           0.866018   0.221887   3.903  9.5e-05 ***
## trestbps    -0.013461   0.012364  -1.089  0.27627    
## chol        -0.008680   0.004755  -1.825  0.06793 .  
## fbs         -0.131686   0.626322  -0.210  0.83347    
## restecg      0.363787   0.407502   0.893  0.37200    
## thalach      0.034392   0.012724   2.703  0.00688 ** 
## exang       -1.048060   0.489902  -2.139  0.03241 *  
## oldpeak     -0.451549   0.258939  -1.744  0.08119 .  
## slope        0.264766   0.411086   0.644  0.51953    
## ca          -0.550719   0.214671  -2.565  0.01031 *  
## thal        -0.958959   0.355134  -2.700  0.00693 ** 
## ---
## Signif. codes:  0 '***' 0.001 '**' 0.01 '*' 0.05 '.' 0.1 ' ' 1
## 
## (Dispersion parameter for binomial family taken to be 1)
## 
##     Null deviance: 292.79  on 212  degrees of freedom
## Residual deviance: 151.91  on 199  degrees of freedom
## AIC: 179.91
## 
## Number of Fisher Scoring iterations: 6
\end{verbatim}

\begin{Shaded}
\begin{Highlighting}[]
\NormalTok{logit\_coef }\OtherTok{\textless{}{-}}\NormalTok{ heart\_logit }\SpecialCharTok{\%\textgreater{}\%}
  \FunctionTok{tidy}\NormalTok{() }\SpecialCharTok{\%\textgreater{}\%} \FunctionTok{select}\NormalTok{(estimate) }\SpecialCharTok{\%\textgreater{}\%} 
  \FunctionTok{rename}\NormalTok{(}\AttributeTok{Logit =}\NormalTok{ estimate)}
\end{Highlighting}
\end{Shaded}

making predictions:

\begin{Shaded}
\begin{Highlighting}[]
\NormalTok{heart\_log\_predict }\OtherTok{\textless{}{-}}\NormalTok{ heart\_test }\SpecialCharTok{\%\textgreater{}\%} 
  \FunctionTok{mutate}\NormalTok{(}
    \AttributeTok{pred=}\FunctionTok{predict}\NormalTok{(heart\_logit,heart\_test,}\AttributeTok{type=}\StringTok{"response"}\NormalTok{)}
\NormalTok{  )}

\NormalTok{heart\_log\_predict}\SpecialCharTok{$}\NormalTok{pred }\SpecialCharTok{\%\textgreater{}\%} \FunctionTok{max}\NormalTok{()}
\end{Highlighting}
\end{Shaded}

\begin{verbatim}
## [1] 0.9974865
\end{verbatim}

\begin{Shaded}
\begin{Highlighting}[]
\NormalTok{heart\_log\_predict}\SpecialCharTok{$}\NormalTok{pred }\SpecialCharTok{\%\textgreater{}\%} \FunctionTok{min}\NormalTok{()}
\end{Highlighting}
\end{Shaded}

\begin{verbatim}
## [1] 0.001112794
\end{verbatim}

calsify predictions for p = 0.9

\begin{Shaded}
\begin{Highlighting}[]
\NormalTok{p }\OtherTok{\textless{}{-}} \FloatTok{0.9}
\NormalTok{heart\_log\_predict }\OtherTok{\textless{}{-}}\NormalTok{ heart\_log\_predict }\SpecialCharTok{\%\textgreater{}\%}
  \FunctionTok{mutate}\NormalTok{(}
    \AttributeTok{pred\_class =} \FunctionTok{if\_else}\NormalTok{(pred }\SpecialCharTok{\textgreater{}}\NormalTok{ p,}\StringTok{"positive"}\NormalTok{,}\StringTok{"negative"}\NormalTok{),}
    \AttributeTok{pred\_class =} \FunctionTok{factor}\NormalTok{(pred\_class),}
    \AttributeTok{target\_fct =} \FunctionTok{if\_else}\NormalTok{(target }\SpecialCharTok{==} \DecValTok{1}\NormalTok{,}\StringTok{"positive"}\NormalTok{,}\StringTok{"negative"}\NormalTok{),}
    \AttributeTok{target\_fct =} \FunctionTok{factor}\NormalTok{(target\_fct)}
\NormalTok{) }
\end{Highlighting}
\end{Shaded}

matrics:

\begin{Shaded}
\begin{Highlighting}[]
\NormalTok{conf\_mtx }\OtherTok{\textless{}{-}}\NormalTok{ heart\_log\_predict  }\SpecialCharTok{\%\textgreater{}\%}
  \FunctionTok{conf\_mat}\NormalTok{(target\_fct,pred\_class)}

\NormalTok{conf\_mtx }\SpecialCharTok{\%\textgreater{}\%} \FunctionTok{summary}\NormalTok{() }\SpecialCharTok{\%\textgreater{}\%} 
  \FunctionTok{pivot\_wider}\NormalTok{(}\AttributeTok{names\_from =}\NormalTok{ .metric, }\AttributeTok{values\_from =}\NormalTok{ .estimate) }\SpecialCharTok{\%\textgreater{}\%} 
  \FunctionTok{select}\NormalTok{(accuracy,spec,sens)}
\end{Highlighting}
\end{Shaded}

\begin{verbatim}
## # A tibble: 1 x 3
##   accuracy  spec  sens
##      <dbl> <dbl> <dbl>
## 1    0.667 0.404 0.953
\end{verbatim}

\hypertarget{regularization}{%
\subsection{Regularization}\label{regularization}}

\hypertarget{penalty}{%
\subsubsection{\texorpdfstring{\emph{Penalty}}{Penalty}}\label{penalty}}

\begin{enumerate}
\def\labelenumi{\arabic{enumi}.}
\tightlist
\item
  The term in the penalty must be in absolute or square terms, beacuse
  we want it to be positive, even if some of the \(\beta\)s are
  negative.
\item
  Since LASSO uses the absolute value in the penalty temm, the optimum
  related to it is usually when some of the \(\beta\) are zero (triangle
  Vs. circle), and therefor it functions as ``variable selection''
\end{enumerate}

\hypertarget{modeling}{%
\subsubsection{\texorpdfstring{\emph{Modeling}}{Modeling}}\label{modeling}}

preparing data for glmnet:

\begin{Shaded}
\begin{Highlighting}[]
\NormalTok{X\_heart }\OtherTok{\textless{}{-}}\NormalTok{ heart\_test }\SpecialCharTok{\%\textgreater{}\%} \FunctionTok{select}\NormalTok{(}\SpecialCharTok{{-}}\NormalTok{target) }\SpecialCharTok{\%\textgreater{}\%}\NormalTok{ as.matrix}
\NormalTok{Y\_heart }\OtherTok{\textless{}{-}}\NormalTok{ heart\_test }\SpecialCharTok{\%\textgreater{}\%} \FunctionTok{select}\NormalTok{(target) }\SpecialCharTok{\%\textgreater{}\%}\NormalTok{ as.matrix}
\end{Highlighting}
\end{Shaded}

Fitting a model

\begin{Shaded}
\begin{Highlighting}[]
\NormalTok{fit\_ridge }\OtherTok{\textless{}{-}} \FunctionTok{glmnet}\NormalTok{(}
  \AttributeTok{x =}\NormalTok{ X\_heart,}
  \AttributeTok{y =}\NormalTok{ Y\_heart,}
  \AttributeTok{alpha =} \DecValTok{0}
\NormalTok{)}
\FunctionTok{plot}\NormalTok{(fit\_ridge,}\AttributeTok{xvar =} \StringTok{"lambda"}\NormalTok{)}
\end{Highlighting}
\end{Shaded}

\includegraphics{ps_2_files/figure-latex/unnamed-chunk-21-1..svg}

\begin{Shaded}
\begin{Highlighting}[]
\NormalTok{cv\_ridge }\OtherTok{\textless{}{-}} \FunctionTok{cv.glmnet}\NormalTok{(}
  \AttributeTok{x =}\NormalTok{ X\_heart,}
  \AttributeTok{y =}\NormalTok{ Y\_heart,}
  \AttributeTok{alpha =} \DecValTok{0} 
\NormalTok{)}
\FunctionTok{plot}\NormalTok{ (cv\_ridge)}
\end{Highlighting}
\end{Shaded}

\includegraphics{ps_2_files/figure-latex/unnamed-chunk-22-1..svg}

Coeffitients:

\begin{Shaded}
\begin{Highlighting}[]
\NormalTok{coef\_1 }\OtherTok{\textless{}{-}} \FunctionTok{coef}\NormalTok{(cv\_ridge, }\AttributeTok{s =} \StringTok{"lambda.min"}\NormalTok{) }\SpecialCharTok{\%\textgreater{}\%}
  \FunctionTok{tidy}\NormalTok{() }\SpecialCharTok{\%\textgreater{}\%}
  \FunctionTok{as\_tibble}\NormalTok{() }\SpecialCharTok{\%\textgreater{}\%}
  \FunctionTok{select}\NormalTok{(value) }\SpecialCharTok{\%\textgreater{}\%}
  \FunctionTok{rename}\NormalTok{(}\AttributeTok{lambda\_min =}\NormalTok{ value)}

\FunctionTok{coef}\NormalTok{(cv\_ridge, }\AttributeTok{s =} \StringTok{"lambda.1se"}\NormalTok{) }\SpecialCharTok{\%\textgreater{}\%}
  \FunctionTok{tidy}\NormalTok{() }\SpecialCharTok{\%\textgreater{}\%}
\NormalTok{  as\_tibble }\SpecialCharTok{\%\textgreater{}\%}
  \FunctionTok{select}\NormalTok{(}\SpecialCharTok{{-}}\NormalTok{column) }\SpecialCharTok{\%\textgreater{}\%} 
  \FunctionTok{rename}\NormalTok{(}\StringTok{"lambda\_1se"} \OtherTok{=}\NormalTok{ value) }\SpecialCharTok{\%\textgreater{}\%} 
  \FunctionTok{bind\_cols}\NormalTok{(coef\_1,logit\_coef)}
\end{Highlighting}
\end{Shaded}

\begin{verbatim}
## # A tibble: 14 x 4
##    row         lambda_1se lambda_min    Logit
##    <chr>            <dbl>      <dbl>    <dbl>
##  1 (Intercept)  0.682       0.769     2.94   
##  2 age         -0.000490    0.00139  -0.0106 
##  3 sex         -0.0725     -0.128    -2.08   
##  4 cp           0.0474      0.0688    0.866  
##  5 trestbps    -0.00133    -0.00221  -0.0135 
##  6 chol         0.0000960   0.000311 -0.00868
##  7 fbs          0.00381     0.00547  -0.132  
##  8 restecg      0.0444      0.0815    0.364  
##  9 thalach      0.00132     0.00112   0.0344 
## 10 exang       -0.111      -0.165    -1.05   
## 11 oldpeak     -0.0367     -0.0442   -0.452  
## 12 slope        0.0761      0.122     0.265  
## 13 ca          -0.0732     -0.138    -0.551  
## 14 thal        -0.0670     -0.104    -0.959
\end{verbatim}

The coefficients are very different from the one form the Logistic
regression, and some of them are really close to zero.

\hypertarget{problem-with-covariats-zero}{%
\subsubsection{problem with covariats =
zero}\label{problem-with-covariats-zero}}

In Econometrics, we are interested in the causual effect of an
observable. Reducing the coeffitiant of a featcher to zero for reasons
of model simplicity might cause us to lose important information.

\hypertarget{prediction}{%
\subsubsection{prediction:}\label{prediction}}

\begin{Shaded}
\begin{Highlighting}[]
\NormalTok{heart\_predict }\OtherTok{\textless{}{-}}\NormalTok{ heart\_test }\SpecialCharTok{\%\textgreater{}\%}
  \FunctionTok{select}\NormalTok{(target) }\SpecialCharTok{\%\textgreater{}\%}
  \FunctionTok{mutate}\NormalTok{(}
    \AttributeTok{log\_pred =} \FunctionTok{predict}\NormalTok{(heart\_logit, }\AttributeTok{newdata =}\NormalTok{ heart\_test),}
    \AttributeTok{min\_lam\_pred =} \FunctionTok{predict}\NormalTok{(cv\_ridge,}
                           \FunctionTok{as.matrix}\NormalTok{(}\FunctionTok{select}\NormalTok{(heart\_test,}\SpecialCharTok{{-}}\NormalTok{target))}
\NormalTok{                           ,}\AttributeTok{s =} \StringTok{"lambda.min"}\NormalTok{)[,}\DecValTok{1}\NormalTok{],}
    \StringTok{"1se\_lam\_pred"} \OtherTok{=} \FunctionTok{predict}\NormalTok{(cv\_ridge,}
                           \FunctionTok{as.matrix}\NormalTok{(}\FunctionTok{select}\NormalTok{(heart\_test,}\SpecialCharTok{{-}}\NormalTok{target))}
\NormalTok{                           ,}\AttributeTok{s =} \StringTok{"lambda.1se"}\NormalTok{)[,}\DecValTok{1}\NormalTok{]}
\NormalTok{  )}
\end{Highlighting}
\end{Shaded}

\emph{The code for this file is on Github, and can be found
\href{https://github.com/zamirD123/ML4ECON_PS_2.git}{here}}

\end{document}
